\documentclass{article}

\pagestyle{empty}

\usepackage{mathtools,amssymb,circuitikz}

\usepackage[margin=0.3in]{geometry}

\begin{document}

    \section*{Bipolo}

        \begin{minipage}[t]{.3\textwidth}
            \vspace{-\baselineskip}

            \subsection*{Utilizzatori}

                \begin{minipage}[t]{.2\textwidth}
                    \vspace{-\baselineskip}

                    \begin{circuitikz}[scale=.75, every node/.style={scale=.75}]
    \draw
    (0,0) to[generic, o-o, i=\(I\), v<=\(V\)] (0,-2.5)
    ;
\end{circuitikz}
                \end{minipage}
                \hfill
                \begin{minipage}[t]{.8\textwidth}
                    \vspace{-\baselineskip}

                    \begin{align*}
                        &V = R \times I \; [\mathrm{V}]\\
                        &P_{Ass} = V \times I \; [\mathrm{W}]\\
                        &P_{Ero} = - V \times I \; [\mathrm{W}]
                    \end{align*}
                \end{minipage}

        \end{minipage}
        \hfill
        \begin{minipage}[t]{.3\textwidth}
            \vspace{-\baselineskip}

            \subsection*{Generatori}
            
                \begin{minipage}[t]{.2\textwidth}
                    \vspace{-\baselineskip}

                    \begin{circuitikz}
    \draw
    (0,0) to[generic, o-o, i=\(I\), v=\(V\)] (0,-2.5)
    ;
\end{circuitikz}
                \end{minipage}
                \hfill
                \begin{minipage}[t]{.8\textwidth}
                    \vspace{-\baselineskip}

                    \begin{align*}
                        &V = - R \times I \; [\mathrm{V}]\\
                        &P_{Ass} = - V \times I \; [\mathrm{W}]\\
                        &P_{Ero} = V \times I \; [\mathrm{W}]
                    \end{align*}
                \end{minipage}

        \end{minipage}
        \hfill
        \begin{minipage}[t]{.3\textwidth}
            \vspace{-\baselineskip}

            \subsection*{Teorema di Tellegen}

                \[
                    \sum_n V_n \times I_n = 0
                \]

        \end{minipage}

    \vspace{-\baselineskip}

    \section*{Partitori}

        \begin{minipage}[t]{.5\textwidth}

            \begin{minipage}[t]{.5\textwidth}
                \vspace{-\baselineskip}

                \centering\begin{circuitikz}[scale=0.6, every node/.style={scale=0.6}]
    \draw
    (0,0) to[american, I, l=\(I\), invert] (0,-2)
    (0,0) to (2,0)
        to[generic, R=$R_1$, i=$I_1$] (2,-2)
        to (0,-2)
    (2,0) to (4,0)
        to[generic, R=$R_2$] (4,-2)
        to (2,-2)
    ;
\end{circuitikz}
                \[
                    I_1 = I \times \frac{R_2}{R_1 + R_2}
                \]
    
            \end{minipage}
            \hfill
            \begin{minipage}[t]{.5\textwidth}
                \vspace{-\baselineskip}
    
                \centering\begin{circuitikz}[scale=0.6, every node/.style={scale=0.6}]
    \draw
    (0,0) to[american, V, l_=\(V\)] (0,-2)
    (2,0) to[generic, R=\(R_2\)] (0,0)
    (2,-2) to[generic, R=\(R_1\), v=\(V_1\)] (2,0)
    (2,-2) to (0,-2)
    ;
\end{circuitikz}
                \[
                    V_ 1 = V \times \frac{R_2}{R_1 + R_2}
                \]

            \end{minipage}

        \end{minipage}
        \hfill
        \begin{minipage}[t]{.5\textwidth}
            \vspace{-\baselineskip}

            \textbf{Nota}: Dovre è presente una maggiore resistenza, sarà\\
            presente una minore intensità di corrente ed una maggiore\\
            tensione.

            \smallskip

            \begin{tabular}{ c | c | c |}
                \cline{2-3}
                & \textbf{Serie} & \textbf{Parallelo}\\
                \hline
                \multicolumn{1}{ | c | }{\textbf{Corrente}} & \(I = I_1 = \dots = I_n\) & \(I = \sum_n I_n\)\\
                \hline
                \multicolumn{1}{ | c | }{\textbf{Tensione}} & \(V = \sum_n V_n\) & \(V = V_1 = \dots = V_n\)\\
                \hline
            \end{tabular}

        \end{minipage}

\end{document}