\documentclass{article}

\pagestyle{empty}

\usepackage{mathtools,amssymb,circuitikz,physics,multicol}

\usepackage[margin=.2in]{geometry}

\begin{document}

    \section*{Bipolo}

        \begin{minipage}[t]{.3\textwidth}
            \vspace{-\baselineskip}

            \subsection*{Utilizzatori}

                \begin{minipage}[t]{.2\textwidth}
                    \vspace{-\baselineskip}

                    \begin{circuitikz}[scale=.75, every node/.style={scale=.75}]
    \draw
    (0,0) to[generic, o-o, i=\(I\), v<=\(V\)] (0,-2.5)
    ;
\end{circuitikz}
                \end{minipage}
                \hfill
                \begin{minipage}[t]{.8\textwidth}
                    \vspace{-1.5\baselineskip}

                    \begin{align*}
                        &V = R \times I \; [\mathrm{V}]\\
                        &P_{Ass} = V \times I \; [\mathrm{W}]\\
                        &P_{Ero} = - V \times I \; [\mathrm{W}]
                    \end{align*}
                \end{minipage}

        \end{minipage}
        \hfill
        \begin{minipage}[t]{.3\textwidth}
            \vspace{-\baselineskip}

            \subsection*{Generatori}
            
                \begin{minipage}[t]{.2\textwidth}
                    \vspace{-\baselineskip}

                    \begin{circuitikz}
    \draw
    (0,0) to[generic, o-o, i=\(I\), v=\(V\)] (0,-2.5)
    ;
\end{circuitikz}
                \end{minipage}
                \hfill
                \begin{minipage}[t]{.8\textwidth}
                    \vspace{-1.5\baselineskip}

                    \begin{align*}
                        &V = - R \times I \; [\mathrm{V}]\\
                        &P_{Ass} = - V \times I \; [\mathrm{W}]\\
                        &P_{Ero} = V \times I \; [\mathrm{W}]
                    \end{align*}
                \end{minipage}

        \end{minipage}
        \hfill
        \begin{minipage}[t]{.3\textwidth}
            \vspace{-\baselineskip}

            \subsection*{Teorema di Tellegen}

                \[
                    \sum V_n \times I_n = 0
                \]

        \end{minipage}

    \vspace{-\baselineskip}
    \section*{Partitori}

        \begin{minipage}[t]{.5\textwidth}

            \begin{minipage}[t]{.5\textwidth}
                \vspace{-\baselineskip}

                \centering\begin{circuitikz}[scale=0.6, every node/.style={scale=0.6}]
    \draw
    (0,0) to[american, I, l=\(I\), invert] (0,-2)
    (0,0) to (2,0)
        to[generic, R=$R_1$, i=$I_1$] (2,-2)
        to (0,-2)
    (2,0) to (4,0)
        to[generic, R=$R_2$] (4,-2)
        to (2,-2)
    ;
\end{circuitikz}
                \[
                    I_1 = I \times \frac{R_2}{R_1 + R_2}
                \]
    
            \end{minipage}
            \hfill
            \begin{minipage}[t]{.5\textwidth}
                \vspace{-\baselineskip}
    
                \centering\begin{circuitikz}[scale=0.6, every node/.style={scale=0.6}]
    \draw
    (0,0) to[american, V, l_=\(V\)] (0,-2)
    (2,0) to[generic, R=\(R_2\)] (0,0)
    (2,-2) to[generic, R=\(R_1\), v=\(V_1\)] (2,0)
    (2,-2) to (0,-2)
    ;
\end{circuitikz}
                \[
                    V_ 1 = V \times \frac{R_2}{R_1 + R_2}
                \]

            \end{minipage}

        \end{minipage}
        \hfill
        \begin{minipage}[t]{.5\textwidth}
            \vspace{-\baselineskip}

            \textbf{Nota}: Dovre è presente una maggiore resistenza, sarà\\
            presente una minore intensità di corrente ed una maggiore\\
            tensione.

            \smallskip

            \begin{tabular}{ c | c | c |}
                \cline{2-3}
                & \textbf{Serie} & \textbf{Parallelo}\\
                \hline
                \multicolumn{1}{ | c | }{\textbf{Corrente}} & \(I = I_1 = \ldots = I_n\) & \(I = \sum I_n\)\\
                \hline
                \multicolumn{1}{ | c | }{\textbf{Tensione}} & \(V = \sum V_n\) & \(V = V_1 = \ldots = V_n\)\\
                \hline
            \end{tabular}

        \end{minipage}

    \medskip

    \hspace{-.6cm} % Per allineare i titoli
    \begin{minipage}[t]{.3\textwidth}

        \section*{Trasformazioni}
            \vspace{-.2cm}

            \subsection*{Stella \(\rightarrow\) triangolo}

                \[
                    G_{12} = \frac{G_1 \times G_2}{\sum G_n}
                \]

            \subsection*{Triangolo \(\rightarrow\) stella}

                \[
                    R_1 = \frac{R_{12} \times R_{13}}{\sum R_n}
                \]

    \end{minipage}
    \hfill
    \begin{minipage}[t]{.7\textwidth}
        
        \section*{Equivalenti}

            \begin{minipage}[t]{.5\textwidth}
                \vspace{-\baselineskip}
                
                \subsection*{Thévenin}

                    \begin{minipage}[t]{.3\textwidth}
                        \vspace{-\baselineskip}

                        \begin{circuitikz}%[scale=0.6, every node/.style={scale=0.6}]
    \draw (0,0) to[short, o-] (1,0)
        to[generic, R=\(R_{Eq}\)] (1,-1.5)
        to[american, V, V=\(V_{Eq}\)] (1,-3)
        to[short, -o] (0,-3);
    \draw (0,-3) [blue] to[open, v^>=\(V_{CA}\), color=blue] (0,0);
\end{circuitikz}

                    \end{minipage}
                    \hfill
                    \begin{minipage}[t]{.7\textwidth}
                        %\vspace{-\baselineskip}

                        \begin{align*}
                            &V_{Eq} = V_{CA}\\
                            &R_{Eq} = \frac{1}{G_{Eq}}
                        \end{align*}

                    \end{minipage}

            \end{minipage}
            \hfill
            \begin{minipage}[t]{.5\textwidth}
                \vspace{-\baselineskip}
                
                \subsection*{Norton}

                    \begin{minipage}[t]{.3\textwidth}
                        \vspace{-\baselineskip}

                        \begin{circuitikz}[scale=.75, every node/.style={scale=.75}]
    \draw (.75,-4) to[short, o-] (.75,-3)
        to[short] (0,-3)
        to[resistor, R=\(G_{Eq}\)] (0,-1)
        to[short] (.75,-1)
        to[short, -o] (.75,0);
    \draw (.75,-3) to[short] (1.5,-3)
        to[american, I, l=\(I_{Eq}\)] (1.5,-1)
        to[short] (.75,-1);
    \draw [blue] (.75,0) to[short, o-] (2.5,0)
        to[short, i=\(I_{CC}\), color=blue] (2.5,-4)
        to[short, -o] (.75,-4);
\end{circuitikz}

                    \end{minipage}
                    \hfill
                    \begin{minipage}[t]{.7\textwidth}
                        %\vspace{-\baselineskip}

                        \begin{align*}
                            &I_{Eq} = I_{CC}\\
                            &R_{Eq} = \frac{1}{G_{Eq}}
                        \end{align*}

                    \end{minipage}

            \end{minipage}

    \end{minipage}

    \vspace{-2\baselineskip}
    \section*{Trasformatore ideale}

        \begin{minipage}[t]{.1\textwidth}
            \vspace{-\baselineskip}

            \begin{circuitikz}[scale=.75, every node/.style={scale=.75}]
    \draw
    (0,0) node[transformer] (T) {}
    node[ocirc] (A) at ([xshift=-1cm]T.A1) {}
    node[ocirc] (B) at ([xshift=-1cm]T.A2) {}
    node[ocirc] (C) at ([xshift=1cm]T.B1) {}
    node[ocirc] (D) at ([xshift=1cm]T.B2) {}
    %node[circ] (E) at ([xshift=0.4cm,yshift=-5pt]T.A1) {}
    %node[circ] (F) at ([xshift=-0.4cm,yshift=-5pt]T.B1) {}
    (T.A1) to[short, -o] (A)
    (T.A2) to [short, -o] (B) 
    (T.B1) to[short, -o] (C)
    (T.B2) to [short, -o] (D)
    ;
    \begin{scope}[shorten >= 10pt,shorten <= 10pt,]
    \draw[<-] (A) -- node[left] {$V_1$} (B);
    \draw[<-] (C) -- node[right] {$V_2$} (D);
    \draw[<->] ([xshift=7pt]T.north west) to[bend left] node[above] {$n:1$} ([xshift=-7pt]T.north east);
    \end{scope}
    \draw[->] ([xshift=-0.9cm,yshift=10pt]T.A1) -- node[above] {$I_1$} +(20pt,0);
    \draw[->] ([xshift=0.9cm,yshift=10pt]T.B1) -- node[above] {$I_2$} +(-20pt,0);
\end{circuitikz}

        \end{minipage}
        \hfill
        \begin{minipage}[t]{.1\textwidth}
            \vspace{-\baselineskip}
            \vspace{.4cm}

            \begin{align*}
                &V_1 = n \times V_2\\
                &I_1 = -\frac{1}{n} \times V_2
            \end{align*}

        \end{minipage}
        \hfill
        \begin{minipage}[t]{.1\textwidth}
            \vspace{-\baselineskip}

            \begin{circuitikz}[scale=.75, every node/.style={scale=.75}]
    \draw
    (0,0) node[transformer] (T) {}
    node[ocirc] (A) at ([xshift=-1cm]T.A1) {}
    node[ocirc] (B) at ([xshift=-1cm]T.A2) {}
    node[ocirc] (C) at ([xshift=1cm]T.B1) {}
    node[ocirc] (D) at ([xshift=1cm]T.B2) {}
    %node[circ] (E) at ([xshift=0.4cm,yshift=-5pt]T.A1) {}
    %node[circ] (F) at ([xshift=-0.4cm,yshift=-5pt]T.B1) {}
    (T.A1) to[short, -o] (A)
    (T.A2) to [short, -o] (B) 
    (T.B1) to[short] (C)
    (T.B2) to [short] (D)
    (C) to[generic, l=\(z\)] (D)
    ;
    \begin{scope}[shorten >= 10pt,shorten <= 10pt,]
    \draw[<-] (A) -- node[left] {$V_1$} (B);
    \draw[<-] ($(C)-(1,0)$) -- node[right] {$V_2$} ($(D)-(1,0)$);
    \draw[<->] ([xshift=7pt]T.north west) to[bend left] node[above] {$n:1$} ([xshift=-7pt]T.north east);
    \end{scope}
    \draw[->] ([xshift=-0.9cm,yshift=10pt]T.A1) -- node[above] {$I_1$} +(20pt,0);
    \draw[->] ([xshift=0.9cm,yshift=10pt]T.B1) -- node[above] {$I_2$} +(-20pt,0);
\end{circuitikz}

        \end{minipage}
        \hfill
        \begin{minipage}[t]{.1\textwidth}
            %\vspace{-\baselineskip}
            \vspace{.3cm}

            \[
                z_{AB} = n^2 \times z \qquad \iff
            \]

        \end{minipage}
        \hfill
        \begin{minipage}[t]{.1\textwidth}
            \vspace{-\baselineskip}
            \vspace{.7cm}

            \begin{circuitikz}[scale=.75, every node/.style={scale=.75}]
    \draw
    (0,0) to[generic, l=\(n^2 \times z\), o-o] (0,-2)
    ;
\end{circuitikz}

        \end{minipage}

    \vspace{-1.5\baselineskip}
    \section*{Doppi bipoli}

        \begin{minipage}[t]{.5\textwidth}
            \vspace{-2\baselineskip}

            \begin{align*}
                &R:
                    \begin{bmatrix}
                        V_1\\
                        V_2
                    \end{bmatrix}
                    =
                    \begin{bmatrix}
                        r_{11} & r_{12}\\
                        r_{21} & r_{22}
                    \end{bmatrix}
                    \begin{bmatrix}
                        I_1\\
                        I_2
                    \end{bmatrix}
                    +
                    \begin{bmatrix}
                        \hat{V_1}\\
                        \hat{V_2}
                    \end{bmatrix}\\
                &G:
                    \begin{bmatrix}
                        I_1\\
                        I_2
                    \end{bmatrix}
                    =
                    \begin{bmatrix}
                        g_{11} & g_{12}\\
                        g_{21} & g_{22}
                    \end{bmatrix}
                    \begin{bmatrix}
                        V_1\\
                        V_2
                    \end{bmatrix}
                    +
                    \begin{bmatrix}
                        \hat{I_1}\\
                        \hat{I_2}
                    \end{bmatrix}
            \end{align*}

            \medskip

            \centering\begin{circuitikz}%[scale=.75, every node/.style={scale=.75}]
    \draw
    (1,.5) to (3,.5)
        to (3,-1.5)
        to (1,-1.5)
        to (1,.5)
    (0,.25) to[short, o-, i=\(I_1\)] (1,.25)
    (4,.25) to[short, o-, i_=\(I_2\)] (3,.25)
    (1,-1.25) to[short, -o, i=\(I_1\)] (0,-1.25)
    (3,-1.25) to[short, -o, i_=\(I_2\)] (4,-1.25)
    (0,-1.25) to[open, v^=\(V_1\)] (0,.25)
    (4,-1.25) to[open, v=\(V_2\)] (4,.25)
    ;
\end{circuitikz}

        \end{minipage}
        \hfill
        \begin{minipage}[t]{.5\textwidth}
            \vspace{-\baselineskip}

            \subsection*{Ibride}
            \vspace{-\baselineskip}

                \begin{align*}
                    &H1:
                        \begin{bmatrix}
                            V_1\\
                            I_2
                        \end{bmatrix}
                        =
                        \begin{bmatrix}
                            h_{11} & h_{12}\\
                            h_{21} & h_{22}
                        \end{bmatrix}
                        \begin{bmatrix}
                            I_1\\
                            V_2
                        \end{bmatrix}
                        +
                        \begin{bmatrix}
                            \hat{V_1}\\
                            \hat{I_2}
                        \end{bmatrix}\\
                    &H2:
                        \begin{bmatrix}
                            I_1\\
                            V_2
                        \end{bmatrix}
                        =
                        \begin{bmatrix}
                            h'_{11} & h'_{12}\\
                            h'_{21} & h'_{22}
                        \end{bmatrix}
                        \begin{bmatrix}
                            V_1\\
                            I_2
                        \end{bmatrix}
                        +
                        \begin{bmatrix}
                            \hat{I_1}\\
                            \hat{V_2}
                        \end{bmatrix}
                \end{align*}

            \subsection*{Trasmissione}
            \vspace{-\baselineskip}

                \begin{align*}
                    &\text{Diretta}:
                        \begin{bmatrix}
                            V_1\\
                            I_1
                        \end{bmatrix}
                        =
                        \begin{bmatrix}
                            t_{11} & t_{12}\\
                            t_{21} & t_{22}
                        \end{bmatrix}
                        \begin{bmatrix}
                            V_2\\
                            -I_2
                        \end{bmatrix}
                        +
                        \begin{bmatrix}
                            \hat{V_1}\\
                            \hat{I_1}
                        \end{bmatrix}\\
                    &\text{Inversa}:
                        \begin{bmatrix}
                            V_2\\
                            -I_2
                        \end{bmatrix}
                        =
                        \begin{bmatrix}
                            t'_{11} & t'_{12}\\
                            t'_{21} & t'_{22}
                        \end{bmatrix}
                        \begin{bmatrix}
                            V_1\\
                            I_1
                        \end{bmatrix}
                        +
                        \begin{bmatrix}
                            \hat{V_2}\\
                            \hat{I_2}
                        \end{bmatrix}
                \end{align*}

        \end{minipage}

        \medskip

        \textbf{Nota}: se le relazioni non vengono trovate risolvendo il circuito, bisogna utilizzare il metodo delle
        prove semplici, spegnendo i generatori secondo necessità , risolvendo i risultanti circuiti.

    \section*{Induttori e generatori}

        \begin{minipage}[t]{.03\textwidth}
            \vspace{-\baselineskip}
            
            \begin{circuitikz}[scale=.75, every node/.style={scale=.75}]
    \draw
    (0,0) to[cute inductor, o-o, i=\(I_L\), v<=\(V_L\)] (0,-2.5)
    ;
\end{circuitikz}

        \end{minipage}
        \hfill
        \begin{minipage}[t]{.2\textwidth}
            \vspace{-\baselineskip}
            
            \begin{align*}
                V_L &= L \times \dv{i_L (t)}{t}\\
                [L] &= [H]
            \end{align*}

        \end{minipage}
        \hfill
        \begin{minipage}[t]{.03\textwidth}
            \vspace{-\baselineskip}
            
            \begin{circuitikz}[scale=.75, every node/.style={scale=.75}]
    \draw
    (0,0) to[capacitor, o-o, i=\(I_C\), v<=\(V_C\)] (0,-2.5)
    ;
\end{circuitikz}

        \end{minipage}
        \hfill
        \begin{minipage}[t]{.2\textwidth}
            \vspace{-\baselineskip}
            
            \begin{align*}
                I_C &= C \times \dv{v_C (t)}{t}\\
                [C] &= [F]
            \end{align*}

        \end{minipage}
        \hfill
        \begin{minipage}[t]{.2\textwidth}
            \vspace{-\baselineskip}
            
            \[
                R = \frac{\overbrace{l}^{\text{lunghezza}}}{\underbrace{s}_{\text{sezione}} \times \underbrace{c}_{\text{conducibilità}}}
            \]

        \end{minipage}

    \newpage

    \section*{Analisi nodale}

        \begin{multicols}{2}
            
            \subsection*{Semplice}
            
                LKC ai nodi con le correnti in funzione dei ponziali di nodo (verso positivo uscente).
                Risolvo poi il sistema risultante.

            \subsection*{Modificata}

                Aggiungo un'equazione per ogni variabile aggiunta non controllabile in tensione.
                Risolvo poi il sistema risultante.

            \vfill\null
            \columnbreak

            \subsection*{Per ispezione}

                \begin{itemize}
                    \item Matrice dei coefficienti:
                        \begin{itemize}
                            \item Diagonale principale posizione \((x,x)\): somma delle conduttanze che arrivano al nodo \(x\).
                            \item Fuori dalla diagonale principale posizione \((i,j)\): la conduttanza tra i nodi \(i\) e \(j\) con segno meno.
                        \end{itemize}
                    \item Vettore dei termini noti riga \(i\): valore del generatore di corrente entrante nel nodo \(i\).
                \end{itemize}

        \end{multicols}

        \textbf{Nota}: Ogni generatore si deve presentare due volte con segno opposto nel vettore dei termini noti od una sola volta
        se collegato al nodo di riferimento.

\end{document}