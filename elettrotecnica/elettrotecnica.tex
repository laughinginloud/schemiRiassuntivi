\documentclass{article}

\pagestyle{empty}

\usepackage{mathtools,amssymb,circuitikz}

\usepackage[margin=0.3in]{geometry}

\begin{document}

    \section*{Bipolo}

        \begin{minipage}[t]{.3\textwidth}
            \vspace{-\baselineskip}

            \subsection*{Utilizzatori}

                \begin{minipage}[t]{.2\textwidth}
                    \vspace{-\baselineskip}

                    \begin{circuitikz}[scale=.75, every node/.style={scale=.75}]
    \draw
    (0,0) to[generic, o-o, i=\(I\), v<=\(V\)] (0,-2.5)
    ;
\end{circuitikz}
                \end{minipage}
                \hfill
                \begin{minipage}[t]{.8\textwidth}
                    \vspace{-\baselineskip}

                    \begin{align*}
                        &V = R \times I \; [\mathrm{V}]\\
                        &P_{Ass} = V \times I \; [\mathrm{W}]\\
                        &P_{Ero} = - V \times I \; [\mathrm{W}]
                    \end{align*}
                \end{minipage}

        \end{minipage}
        \hfill
        \begin{minipage}[t]{.3\textwidth}
            \vspace{-\baselineskip}

            \subsection*{Generatori}
            
                \begin{minipage}[t]{.2\textwidth}
                    \vspace{-\baselineskip}

                    \begin{circuitikz}
    \draw
    (0,0) to[generic, o-o, i=\(I\), v=\(V\)] (0,-2.5)
    ;
\end{circuitikz}
                \end{minipage}
                \hfill
                \begin{minipage}[t]{.8\textwidth}
                    \vspace{-\baselineskip}

                    \begin{align*}
                        &V = - R \times I \; [\mathrm{V}]\\
                        &P_{Ass} = - V \times I \; [\mathrm{W}]\\
                        &P_{Ero} = V \times I \; [\mathrm{W}]
                    \end{align*}
                \end{minipage}

        \end{minipage}
        \hfill
        \begin{minipage}[t]{.3\textwidth}
            \vspace{-\baselineskip}

            \subsection*{Teorema di Tellegen}

                \[
                    \sum V_n \times I_n = 0
                \]

        \end{minipage}

    \vspace{-\baselineskip}

    \section*{Partitori}

        \begin{minipage}[t]{.5\textwidth}

            \begin{minipage}[t]{.5\textwidth}
                \vspace{-\baselineskip}

                \centering\begin{circuitikz}[scale=0.6, every node/.style={scale=0.6}]
    \draw
    (0,0) to[american, I, l=\(I\), invert] (0,-2)
    (0,0) to (2,0)
        to[generic, R=$R_1$, i=$I_1$] (2,-2)
        to (0,-2)
    (2,0) to (4,0)
        to[generic, R=$R_2$] (4,-2)
        to (2,-2)
    ;
\end{circuitikz}
                \[
                    I_1 = I \times \frac{R_2}{R_1 + R_2}
                \]
    
            \end{minipage}
            \hfill
            \begin{minipage}[t]{.5\textwidth}
                \vspace{-\baselineskip}
    
                \centering\begin{circuitikz}[scale=0.6, every node/.style={scale=0.6}]
    \draw
    (0,0) to[american, V, l_=\(V\)] (0,-2)
    (2,0) to[generic, R=\(R_2\)] (0,0)
    (2,-2) to[generic, R=\(R_1\), v=\(V_1\)] (2,0)
    (2,-2) to (0,-2)
    ;
\end{circuitikz}
                \[
                    V_ 1 = V \times \frac{R_2}{R_1 + R_2}
                \]

            \end{minipage}

        \end{minipage}
        \hfill
        \begin{minipage}[t]{.5\textwidth}
            \vspace{-\baselineskip}

            \textbf{Nota}: Dovre è presente una maggiore resistenza, sarà\\
            presente una minore intensità di corrente ed una maggiore\\
            tensione.

            \smallskip

            \begin{tabular}{ c | c | c |}
                \cline{2-3}
                & \textbf{Serie} & \textbf{Parallelo}\\
                \hline
                \multicolumn{1}{ | c | }{\textbf{Corrente}} & \(I = I_1 = \ldots = I_n\) & \(I = \sum I_n\)\\
                \hline
                \multicolumn{1}{ | c | }{\textbf{Tensione}} & \(V = \sum V_n\) & \(V = V_1 = \ldots = V_n\)\\
                \hline
            \end{tabular}

        \end{minipage}

    \medskip

    \hspace{-.6cm} % Per allineare i titoli
    \begin{minipage}[t]{.3\textwidth}
        
        \section*{Trasformazioni}

            \subsection*{Stella \(\rightarrow\) triangolo}

                \[
                    G_{12} = \frac{G_1 \times G_2}{\sum G_n}
                \]

                \subsection*{Triangolo \(\rightarrow\) stella}

                \[
                    R_1 = \frac{R_{12} \times R_{13}}{\sum R_n}
                \]
    \end{minipage}
    \hfill
    \begin{minipage}[t]{.7\textwidth}
        
        \section*{Equivalenti}

            \begin{minipage}[t]{.5\textwidth}
                
                \subsection*{Thévenin}

                    \begin{minipage}[t]{.3\textwidth}
                        \vspace{-\baselineskip}

                        \begin{circuitikz}%[scale=0.6, every node/.style={scale=0.6}]
    \draw (0,0) to[short, o-] (1,0)
        to[generic, R=\(R_{Eq}\)] (1,-1.5)
        to[american, V, V=\(V_{Eq}\)] (1,-3)
        to[short, -o] (0,-3);
    \draw (0,-3) [blue] to[open, v^>=\(V_{CA}\), color=blue] (0,0);
\end{circuitikz}

                    \end{minipage}
                    \hfill
                    \begin{minipage}[t]{.7\textwidth}
                        %\vspace{-\baselineskip}

                        \begin{align*}
                            &V_{Eq} = V_{CA}\\
                            &R_{Eq} = \frac{1}{G_{Eq}}
                        \end{align*}

                    \end{minipage}

            \end{minipage}
            \hfill
            \begin{minipage}[t]{.5\textwidth}
                
                \subsection*{Norton}

                    \begin{minipage}[t]{.3\textwidth}
                        \vspace{-\baselineskip}

                        \begin{circuitikz}[scale=.75, every node/.style={scale=.75}]
    \draw (.75,-4) to[short, o-] (.75,-3)
        to[short] (0,-3)
        to[resistor, R=\(G_{Eq}\)] (0,-1)
        to[short] (.75,-1)
        to[short, -o] (.75,0);
    \draw (.75,-3) to[short] (1.5,-3)
        to[american, I, l=\(I_{Eq}\)] (1.5,-1)
        to[short] (.75,-1);
    \draw [blue] (.75,0) to[short, o-] (2.5,0)
        to[short, i=\(I_{CC}\), color=blue] (2.5,-4)
        to[short, -o] (.75,-4);
\end{circuitikz}

                    \end{minipage}
                    \hfill
                    \begin{minipage}[t]{.7\textwidth}
                        %\vspace{-\baselineskip}

                        \begin{align*}
                            &I_{Eq} = I_{CC}\\
                            &R_{Eq} = \frac{1}{G_{Eq}}
                        \end{align*}

                    \end{minipage}

            \end{minipage}

    \end{minipage}

\end{document}