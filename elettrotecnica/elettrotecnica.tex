\documentclass[10pt]{article}

\pagestyle{empty}

\usepackage{mathtools,amssymb,circuitikz,physics,multicol}

\usepackage[margin=.2in]{geometry}

\renewcommand{\deg}{^{\circ}}
\newcommand{\unit}[1]{\left[\mathbf{#1}\right]}

\begin{document}

    \subsection*{Bipolo}

        \begin{minipage}[t]{.3\textwidth}
            \vspace{-\baselineskip}

            \subsubsection*{Utilizzatori}

                \begin{minipage}[t]{.2\textwidth}
                    \vspace{-\baselineskip}

                    \begin{circuitikz}[scale=.75, every node/.style={scale=.75}]
    \draw
    (0,0) to[generic, o-o, i=\(I\), v<=\(V\)] (0,-2.5)
    ;
\end{circuitikz}
                \end{minipage}
                \hfill
                \begin{minipage}[t]{.8\textwidth}
                    \vspace{-1.5\baselineskip}

                    \begin{align*}
                        V &= R \times I \; \unit{V}\\
                        P_{Ass} &= V \times I \; \unit{W}\\
                        P_{Ero} &= - V \times I \; \unit{W}
                    \end{align*}
                \end{minipage}

        \end{minipage}
        \hfill
        \begin{minipage}[t]{.3\textwidth}
            \vspace{-\baselineskip}

            \subsubsection*{Generatori}
            
                \begin{minipage}[t]{.2\textwidth}
                    \vspace{-\baselineskip}

                    \begin{circuitikz}
    \draw
    (0,0) to[generic, o-o, i=\(I\), v=\(V\)] (0,-2.5)
    ;
\end{circuitikz}
                \end{minipage}
                \hfill
                \begin{minipage}[t]{.8\textwidth}
                    \vspace{-1.5\baselineskip}

                    \begin{align*}
                        V &= - R \times I \; \unit{V}\\
                        P_{Ass} &= - V \times I \; \unit{W}\\
                        P_{Ero} &= V \times I \; \unit{W}
                    \end{align*}
                \end{minipage}

        \end{minipage}
        \hfill
        \begin{minipage}[t]{.3\textwidth}
            \vspace{-\baselineskip}

            \subsubsection*{Teorema di Tellegen}

                \[
                    \sum V_n \times I_n = 0
                \]

            \vspace{-\baselineskip}

            %\vspace{-1.3cm}

            %\subsubsection*{}
            %\[
            %    \Phi_B (t) = A(t) \times B(t)
            %\]

        \end{minipage}

    \vspace{-\baselineskip}
    \subsection*{Partitori}

        \begin{minipage}[t]{.55\textwidth}

            \begin{minipage}[t]{.5\textwidth}
                \vspace{-\baselineskip}

                \centering\begin{circuitikz}[scale=0.6, every node/.style={scale=0.6}]
    \draw
    (0,0) to[american, I, l=\(I\), invert] (0,-2)
    (0,0) to (2,0)
        to[generic, R=$R_1$, i=$I_1$] (2,-2)
        to (0,-2)
    (2,0) to (4,0)
        to[generic, R=$R_2$] (4,-2)
        to (2,-2)
    ;
\end{circuitikz}
                \[
                    I_1 = I \times \frac{R_2}{R_1 + R_2} = I \times \frac{G_1}{\sum G_N}
                \]
    
            \end{minipage}
            \hfill
            \begin{minipage}[t]{.5\textwidth}
                \vspace{-\baselineskip}
    
                \centering\begin{circuitikz}[scale=0.6, every node/.style={scale=0.6}]
    \draw
    (0,0) to[american, V, l_=\(V\)] (0,-2)
    (2,0) to[generic, R=\(R_2\)] (0,0)
    (2,-2) to[generic, R=\(R_1\), v=\(V_1\)] (2,0)
    (2,-2) to (0,-2)
    ;
\end{circuitikz}
                \[
                    V_ 1 = V \times \frac{R_1}{R_1 + R_2} = V \times \frac{R_1}{\sum R_n}
                \]

            \end{minipage}

        \end{minipage}
        \hfill
        \begin{minipage}[t]{.45\textwidth}
            \vspace{-\baselineskip}

            \textbf{Nota}: Dove è presente una maggiore resistenza, sarà\\
            presente una minore intensità di corrente ed una maggiore\\
            tensione.

            \smallskip

            \begin{tabular}{ c | c | c |}
                \cline{2-3}
                & \textbf{Serie} & \textbf{Parallelo}\\
                \hline
                \multicolumn{1}{ | c | }{\textbf{Corrente}} & \(I = I_1 = \ldots = I_n\) & \(I = \sum I_n\)\\
                \hline
                \multicolumn{1}{ | c | }{\textbf{Tensione}} & \(V = \sum V_n\) & \(V = V_1 = \ldots = V_n\)\\
                \hline
            \end{tabular}

        \end{minipage}

    \medskip

    \hspace{-.6cm} % Per allineare i titoli
    \begin{minipage}[t]{.3\textwidth}

        \subsection*{Trasformazioni}
            \vspace{-.2cm}

            \subsubsection*{Stella \(\rightarrow\) triangolo}

                \[
                    G_{12} = \frac{G_1 \times G_2}{\sum G_n}
                \]

            \subsubsection*{Triangolo \(\rightarrow\) stella}

                \[
                    R_1 = \frac{R_{12} \times R_{13}}{\sum R_n}
                \]

    \end{minipage}
    \hfill
    \begin{minipage}[t]{.7\textwidth}
        
        \subsection*{Equivalenti}

            \begin{minipage}[t]{.5\textwidth}
                \vspace{-\baselineskip}
                
                \subsubsection*{Thévenin}

                    \begin{minipage}[t]{.3\textwidth}
                        \vspace{-\baselineskip}

                        \begin{circuitikz}%[scale=0.6, every node/.style={scale=0.6}]
    \draw (0,0) to[short, o-] (1,0)
        to[generic, R=\(R_{Eq}\)] (1,-1.5)
        to[american, V, V=\(V_{Eq}\)] (1,-3)
        to[short, -o] (0,-3);
    \draw (0,-3) [blue] to[open, v^>=\(V_{CA}\), color=blue] (0,0);
\end{circuitikz}

                    \end{minipage}
                    \hfill
                    \begin{minipage}[t]{.7\textwidth}
                        %\vspace{-\baselineskip}

                        \begin{align*}
                            V_{Eq} &= V_{CA}\\
                            R_{Eq} &= \frac{1}{G_{Eq}}
                        \end{align*}

                    \end{minipage}

            \end{minipage}
            \hfill
            \begin{minipage}[t]{.5\textwidth}
                \vspace{-\baselineskip}
                
                \subsubsection*{Norton}

                    \begin{minipage}[t]{.3\textwidth}
                        \vspace{-\baselineskip}

                        \begin{circuitikz}[scale=.75, every node/.style={scale=.75}]
    \draw (.75,-4) to[short, o-] (.75,-3)
        to[short] (0,-3)
        to[resistor, R=\(G_{Eq}\)] (0,-1)
        to[short] (.75,-1)
        to[short, -o] (.75,0);
    \draw (.75,-3) to[short] (1.5,-3)
        to[american, I, l=\(I_{Eq}\)] (1.5,-1)
        to[short] (.75,-1);
    \draw [blue] (.75,0) to[short, o-] (2.5,0)
        to[short, i=\(I_{CC}\), color=blue] (2.5,-4)
        to[short, -o] (.75,-4);
\end{circuitikz}

                    \end{minipage}
                    \hfill
                    \begin{minipage}[t]{.7\textwidth}
                        %\vspace{-\baselineskip}

                        \begin{align*}
                            I_{Eq} &= I_{CC}\\
                            G_{Eq} &= \frac{1}{R_{Eq}}
                        \end{align*}

                    \end{minipage}

            \end{minipage}

    \end{minipage}

    \vspace{-2\baselineskip}
    \subsection*{Trasformatore ideale}

        \begin{minipage}[t]{.1\textwidth}
            \vspace{-\baselineskip}

            \begin{circuitikz}[scale=.75, every node/.style={scale=.75}]
    \draw
    (0,0) node[transformer] (T) {}
    node[ocirc] (A) at ([xshift=-1cm]T.A1) {}
    node[ocirc] (B) at ([xshift=-1cm]T.A2) {}
    node[ocirc] (C) at ([xshift=1cm]T.B1) {}
    node[ocirc] (D) at ([xshift=1cm]T.B2) {}
    %node[circ] (E) at ([xshift=0.4cm,yshift=-5pt]T.A1) {}
    %node[circ] (F) at ([xshift=-0.4cm,yshift=-5pt]T.B1) {}
    (T.A1) to[short, -o] (A)
    (T.A2) to [short, -o] (B) 
    (T.B1) to[short, -o] (C)
    (T.B2) to [short, -o] (D)
    ;
    \begin{scope}[shorten >= 10pt,shorten <= 10pt,]
    \draw[<-] (A) -- node[left] {$V_1$} (B);
    \draw[<-] (C) -- node[right] {$V_2$} (D);
    \draw[<->] ([xshift=7pt]T.north west) to[bend left] node[above] {$n:1$} ([xshift=-7pt]T.north east);
    \end{scope}
    \draw[->] ([xshift=-0.9cm,yshift=10pt]T.A1) -- node[above] {$I_1$} +(20pt,0);
    \draw[->] ([xshift=0.9cm,yshift=10pt]T.B1) -- node[above] {$I_2$} +(-20pt,0);
\end{circuitikz}

        \end{minipage}
        \hfill
        \begin{minipage}[t]{.1\textwidth}
            \vspace{-\baselineskip}
            \vspace{.4cm}

            \begin{align*}
                V_1 &= n \times V_2\\
                I_1 &= -\frac{1}{n} \times I_2
            \end{align*}

        \end{minipage}
        \hfill
        \begin{minipage}[t]{.1\textwidth}
            \vspace{-\baselineskip}

            \begin{circuitikz}[scale=.75, every node/.style={scale=.75}]
    \draw
    (0,0) node[transformer] (T) {}
    node[ocirc] (A) at ([xshift=-1cm]T.A1) {}
    node[ocirc] (B) at ([xshift=-1cm]T.A2) {}
    node[ocirc] (C) at ([xshift=1cm]T.B1) {}
    node[ocirc] (D) at ([xshift=1cm]T.B2) {}
    %node[circ] (E) at ([xshift=0.4cm,yshift=-5pt]T.A1) {}
    %node[circ] (F) at ([xshift=-0.4cm,yshift=-5pt]T.B1) {}
    (T.A1) to[short, -o] (A)
    (T.A2) to [short, -o] (B) 
    (T.B1) to[short] (C)
    (T.B2) to [short] (D)
    (C) to[generic, l=\(z\)] (D)
    ;
    \begin{scope}[shorten >= 10pt,shorten <= 10pt,]
    \draw[<-] (A) -- node[left] {$V_1$} (B);
    \draw[<-] ($(C)-(1,0)$) -- node[right] {$V_2$} ($(D)-(1,0)$);
    \draw[<->] ([xshift=7pt]T.north west) to[bend left] node[above] {$n:1$} ([xshift=-7pt]T.north east);
    \end{scope}
    \draw[->] ([xshift=-0.9cm,yshift=10pt]T.A1) -- node[above] {$I_1$} +(20pt,0);
    \draw[->] ([xshift=0.9cm,yshift=10pt]T.B1) -- node[above] {$I_2$} +(-20pt,0);
\end{circuitikz}

        \end{minipage}
        \hfill
        \begin{minipage}[t]{.1\textwidth}
            %\vspace{-\baselineskip}
            \vspace{.3cm}

            \[
                z_{AB} = n^2 \times z \qquad \iff
            \]

        \end{minipage}
        \hfill
        \begin{minipage}[t]{.1\textwidth}
            \vspace{-\baselineskip}
            \vspace{.7cm}

            \begin{circuitikz}[scale=.75, every node/.style={scale=.75}]
    \draw
    (0,0) to[generic, l=\(n^2 \times z\), o-o] (0,-2)
    ;
\end{circuitikz}

        \end{minipage}

    \vspace{-1.5\baselineskip}
    \subsection*{Doppi bipoli}

        \begin{minipage}[t]{.5\textwidth}
            \vspace{-2\baselineskip}

            \begin{align*}
                &R:
                    \begin{bmatrix}
                        V_1\\
                        V_2
                    \end{bmatrix}
                    =
                    \begin{bmatrix}
                        r_{11} & r_{12}\\
                        r_{21} & r_{22}
                    \end{bmatrix}
                    \begin{bmatrix}
                        I_1\\
                        I_2
                    \end{bmatrix}
                    +
                    \begin{bmatrix}
                        \hat{V_1}\\
                        \hat{V_2}
                    \end{bmatrix}\\
                &G:
                    \begin{bmatrix}
                        I_1\\
                        I_2
                    \end{bmatrix}
                    =
                    \begin{bmatrix}
                        g_{11} & g_{12}\\
                        g_{21} & g_{22}
                    \end{bmatrix}
                    \begin{bmatrix}
                        V_1\\
                        V_2
                    \end{bmatrix}
                    +
                    \begin{bmatrix}
                        \hat{I_1}\\
                        \hat{I_2}
                    \end{bmatrix}
            \end{align*}

            \medskip

            \centering\begin{circuitikz}%[scale=.75, every node/.style={scale=.75}]
    \draw
    (1,.5) to (3,.5)
        to (3,-1.5)
        to (1,-1.5)
        to (1,.5)
    (0,.25) to[short, o-, i=\(I_1\)] (1,.25)
    (4,.25) to[short, o-, i_=\(I_2\)] (3,.25)
    (1,-1.25) to[short, -o, i=\(I_1\)] (0,-1.25)
    (3,-1.25) to[short, -o, i_=\(I_2\)] (4,-1.25)
    (0,-1.25) to[open, v^=\(V_1\)] (0,.25)
    (4,-1.25) to[open, v=\(V_2\)] (4,.25)
    ;
\end{circuitikz}

        \end{minipage}
        \hfill
        \begin{minipage}[t]{.5\textwidth}
            \vspace{-\baselineskip}

            \subsubsection*{Ibride}
            \vspace{-\baselineskip}

                \begin{align*}
                    &\text{Diretta}:
                        \begin{bmatrix}
                            V_1\\
                            I_2
                        \end{bmatrix}
                        =
                        \begin{bmatrix}
                            h_{11} & h_{12}\\
                            h_{21} & h_{22}
                        \end{bmatrix}
                        \begin{bmatrix}
                            I_1\\
                            V_2
                        \end{bmatrix}
                        +
                        \begin{bmatrix}
                            \hat{V_1}\\
                            \hat{I_2}
                        \end{bmatrix}\\
                    &\text{Inversa}:
                        \begin{bmatrix}
                            I_1\\
                            V_2
                        \end{bmatrix}
                        =
                        \begin{bmatrix}
                            h'_{11} & h'_{12}\\
                            h'_{21} & h'_{22}
                        \end{bmatrix}
                        \begin{bmatrix}
                            V_1\\
                            I_2
                        \end{bmatrix}
                        +
                        \begin{bmatrix}
                            \hat{I_1}\\
                            \hat{V_2}
                        \end{bmatrix}
                \end{align*}

            \subsubsection*{Trasmissione}
            \vspace{-\baselineskip}

                \begin{align*}
                    &\text{Diretta}:
                        \begin{bmatrix}
                            V_1\\
                            I_1
                        \end{bmatrix}
                        =
                        \begin{bmatrix}
                            t_{11} & t_{12}\\
                            t_{21} & t_{22}
                        \end{bmatrix}
                        \begin{bmatrix}
                            V_2\\
                            -I_2
                        \end{bmatrix}
                        +
                        \begin{bmatrix}
                            \hat{V_1}\\
                            \hat{I_1}
                        \end{bmatrix}\\
                    &\text{Inversa}:
                        \begin{bmatrix}
                            V_2\\
                            I_2
                        \end{bmatrix}
                        =
                        \begin{bmatrix}
                            t'_{11} & t'_{12}\\
                            t'_{21} & t'_{22}
                        \end{bmatrix}
                        \begin{bmatrix}
                            V_1\\
                            -I_1
                        \end{bmatrix}
                        +
                        \begin{bmatrix}
                            \hat{V_2}\\
                            \hat{I_2}
                        \end{bmatrix}
                \end{align*}

        \end{minipage}

        \medskip

        \textbf{Nota}: se le relazioni non vengono trovate risolvendo il circuito, bisogna utilizzare il metodo delle
        prove semplici, spegnendo i generatori secondo necessità, risolvendo i risultanti circuiti. Per verificare
        l'esistenza di altre formulazioni, verificare che il determinante della matrice dei coefficienti delle
        variabili controllate sia diverso da zero.

    \vspace{-\baselineskip}
    \subsection*{Induttori e condensatori}

        \begin{minipage}[t]{.03\textwidth}
            \vspace{-\baselineskip}
            
            \begin{circuitikz}[scale=.75, every node/.style={scale=.75}]
    \draw
    (0,0) to[cute inductor, o-o, i=\(I_L\), v<=\(V_L\)] (0,-2.5)
    ;
\end{circuitikz}

        \end{minipage}
        %\hfill
        \begin{minipage}[t]{.2\textwidth}
            \vspace{-\baselineskip}
            
            \begin{align*}
                V_L &= L \times \dv{i_L (t)}{t}\\
                [L] &= \unit{H}
            \end{align*}

        \end{minipage}
        %\hfill
        \begin{minipage}[t]{.03\textwidth}
            \vspace{-\baselineskip}
            
            \begin{circuitikz}[scale=.75, every node/.style={scale=.75}]
    \draw
    (0,0) to[capacitor, o-o, i=\(I_C\), v<=\(V_C\)] (0,-2.5)
    ;
\end{circuitikz}

        \end{minipage}
        %\hfill
        \begin{minipage}[t]{.2\textwidth}
            \vspace{-\baselineskip}
            
            \begin{align*}
                \qquad I_C &= C \times \dv{v_C (t)}{t}\\
                \qquad [C] &= \unit{F}\\
            \end{align*}

        \end{minipage}
        %\hfill
        \begin{minipage}[t]{.2\textwidth}
            \vspace{-\baselineskip}
            
            \[
                R = \frac{\overbrace{l}^{\text{lunghezza}}}{\underbrace{s}_{\text{sezione}} \times \underbrace{c}_{\text{conducibilità}}}
            \]

        \end{minipage}
        \begin{minipage}[t]{.3\textwidth}
            \vspace{-\baselineskip}
            
            \subsubsection*{Equazione di stato}
            \vspace{-.23cm}

                \begin{enumerate}
                    \item Trovare la duale della variabile di stato in funzione di quest'ultima
                    \item Sostituire la duale con la relazione costituente
                \end{enumerate}

        \end{minipage}

    \hspace{-.65cm}
    \begin{minipage}[t]{.5\textwidth}
        \subsection*{Generatori trifase}
    
            \[
                \abs{\overline{V}_L} = \sqrt{3} V_{Fase} \qquad \abs{\overline{I}_L} = \sqrt{3} I_{Fase} \qquad V_{Fase} = \abs{\overline{E}_1} \qquad I_{Fase} = \abs{\overline{I}_{f31}}
            \]
    \end{minipage}
    \hfill
    \begin{minipage}[t]{.4\textwidth}
        \subsection*{Frequenza di risonanza}

            \[
                \omega = \frac{1}{\sqrt{LC}}
            \]
    \end{minipage}

    \hspace{-.65cm}
    \begin{minipage}[t]{.45\textwidth}
        \subsection*{Induttori accoppiati in serie e parallelo}
    
            \[
                \text{Serie: } L_{Eq} = L_1 + L_2 \pm 2M \qquad\qquad \text{Parallelo: } L_{Eq} = \frac{L_1 L_2 - 2M}{L_1 + L_2 \mp 2M}
            \]
    \end{minipage}
    \hfill
    \begin{minipage}[t]{.45\textwidth}
        \subsection*{Rifasamento}

            \[
                C = \frac{P \times \tan \varphi - P \times \tan \varphi_{Rifasato}}{\omega \times V^2} \qquad \text{\textbf{N.B.}: }Q = P \times \tan \varphi
            \]
    \end{minipage}

    \newpage

    \subsection*{Analisi nodale}
    \vspace{-1.5\baselineskip}

        \begin{multicols}{2}
            
            \subsubsection*{Semplice}
            
                LKC ai nodi con le correnti in funzione dei ponziali di nodo (verso positivo uscente).
                Risolvo poi il sistema risultante.

            \subsubsection*{Modificata}

                Aggiungo un'equazione per ogni variabile aggiunta non controllabile in tensione.
                Risolvo poi il sistema risultante.

            \vfill\null
            \columnbreak

            \subsubsection*{Per ispezione}

                \begin{itemize}
                    \item Matrice dei coefficienti:
                        \begin{itemize}
                            \item Diagonale principale posizione \((x,x)\): somma delle conduttanze che arrivano al nodo \(x\).
                            \item Fuori dalla diagonale principale posizione \((i,j)\): la conduttanza tra i nodi \(i\) e \(j\) con segno meno.
                        \end{itemize}
                    \item Vettore dei termini noti riga \(i\): valore del generatore di corrente entrante nel nodo \(i\).
                \end{itemize}

        \end{multicols}

        \vspace{-\baselineskip}
        \textbf{Nota}: Ogni generatore si deve presentare due volte con segno opposto nel vettore dei termini noti od una sola volta
        se collegato al nodo di riferimento.

    \vspace{-\baselineskip}
    \subsection*{Regime alternato sinusoidale}
    \vspace{-1.5\baselineskip}

        \begin{gather*}
            v(t) = \underbrace{A}_{\text{ampiezza}} \times \cos (\underbrace{\omega}_{\text{pulsazione}} t + \underbrace{\varphi}_{\text{fase}}) \iff \underbrace{\overline{V}}_{\text{fasore}} = \underbrace{A}_{\text{ampiezza}} \times e^{j \overbrace{\varphi}^{\text{fase}}} = a + jb\\
            \underbrace{\omega}_{\text{pulsazione}} = 2 \pi \underbrace{\nu}_{\text{frequenza}} \qquad A = \sqrt{a^2 + b^2} \qquad \varphi = \arctan{\frac{b}{a}} \qquad \text{\textbf{Nota}: attenzione al quadrante.} \qquad \text{\textbf{Nota}: } \overline{V} \in \mathbb{C}.\\
            \underbrace{Z}_{\text{impedenza}} = \underbrace{R}_{\text{resistenza}} + j \underbrace{X}_{\text{reattanza}} \qquad \underbrace{Y}_{\text{ammettenza}} = \underbrace{G}_{\text{conduttanza}} + j \underbrace{B}_{\text{suscettanza}} \qquad \measuredangle Z = - \measuredangle Y
        \end{gather*}

    \vspace{-1.5\baselineskip}
    \subsection*{Resistori, condensatori ed induttori in RAS}
    \vspace{-1.5\baselineskip}

        \begin{multicols}{4}
            
            \subsubsection*{Resisori}

                \begin{align*}
                    Z_R &= R\\
                    Y_R &= \frac{1}{R}
                \end{align*}

            \vfill\null
            \columnbreak

            \subsubsection*{Condensatori}

                \begin{align*}
                    Z_C &= \frac{1}{j \omega C} = - \frac{1}{\omega C} \times j\\
                    Y_C &= j \omega C
                \end{align*}

            \vfill\null
            \columnbreak

            \subsubsection*{Induttori}

                \begin{align*}
                    Z_L &= j \omega L\\
                    Y_L &= \frac{1}{j \omega L} = -\frac{1}{\omega L} \times j
                \end{align*}

            \vfill\null
            \columnbreak

            \noindent L'\textbf{immettenza} è un\\termine generico per indicare l'impedenza o l'ammettenza.

        \end{multicols}

    \vspace{-3\baselineskip}
    \subsection*{Potenza in RAS}
    \vspace{-1.5\baselineskip}

        \begin{gather*}
            \underbrace{S}_{\text{potenza complessa [VA]}} = \underbrace{P}_{\text{potenza attiva [W]}} + j \underbrace{Q}_{\text{potenza reattiva [VAR]}} \qquad
                S =
                    \begin{cases}
                        \overline{V}_{Eff} \times \overline{I}^*_{Eff}\\
                        \frac{1}{2} \times \overline{V} \times \overline{I}^*
                    \end{cases}
                \quad \overline{V}_{Eff} = \frac{\overline{V}}{\sqrt{2}} \quad \overline{I}_{Eff} = \frac{\overline{I}}{\sqrt{2}}\\
                S = \abs{S} \times \underbrace{\cos \varphi}_{\text{fattore di potenza}} + j \abs{S} \times \sin \varphi \qquad \cos \varphi = \frac{P}{\abs{S}} \qquad \text{Notiamo che } \measuredangle \, \overline{I} = - \measuredangle \, \overline{I}^* \text{, quindi } \varphi = \varphi_{Tensione} - \varphi_{Corrente}\\
                \qquad\quad\textbf{Massimo trasferimento di potenza: } Z_{Sorgente} = Z^*_{Carico} \qquad \textbf{Potenza apparente: } \abs{S} = V_{Eff} \times I_{Eff} \text{ [VA]}
            \end{gather*}

    \vspace{-2\baselineskip}
    \subsection*{Bipoli passivi}
    \vspace{-.5\baselineskip}

        \begin{minipage}[t]{.45\textwidth}
            \begin{itemize}
                \item Bipoli passivi: \(R \geqslant 0,\, G \geqslant 0,\, P \geqslant 0,\, -90\deg \leqslant \varphi \leqslant 90\deg\)
                \item Bipoli resistivi: \(X=B=0,\, Q=0,\, \varphi=0\deg\)
                \item Bipoli reattivi: \(R=G=0,\, P=0,\, \varphi=\pm90\deg\)
            \end{itemize}
        \end{minipage}
        \hfill
        \begin{minipage}[t]{.55\textwidth}
            \begin{itemize}
                \item Bipoli induttivi: \(X > 0,\, B<0,\, Q>0,\, 0\deg < \varphi \leqslant 90\deg\), ritardo
                \item Bipoli capacitivi: \(X<0,\, B>0,\, Q<0,\, -90\deg \leqslant \varphi < 0\deg\), anticipo
            \end{itemize}
        \end{minipage}

    \vspace{-\baselineskip}
    \subsection*{Induttori mutuamente accoppiati}

        \[
            \text{Tempo:}
            \left\{
                \begin{aligned}
                    v_1 (t) &= L_1 \dv{i_1 (t)}{t} + M \dv{i_2 (t)}{t}\\
                    v_2 (t) &= M \dv{i_1 (t)}{t} + L_2 \dv{i_2 (t)}{t}
                \end{aligned}
            \right.
            \iff
            \text{Fasori:}
            \left\{
                \begin{aligned}
                    \overline{V}_1 &= j \omega L_1 \overline{I}_1 + j \omega M \overline{I}_2\\
                    \overline{V}_2 &= j \omega M \overline{I}_1 + j \omega L_2 \overline{I}_2
                \end{aligned}
            \right.
            \qquad\qquad
            \underbrace{k}_{\text{coefficiente di accoppiamento}} = \frac{\abs{M}}{\sqrt{L_1 L_2}}
        \]

    \vspace{-1.5\baselineskip}
    \subsection*{Transitorio}
    \vspace{-1.5\baselineskip}

        \begin{multicols}{3}
            
            Esponenziale
            \begin{itemize}
                \item \(i_L (t) = I_{L \infty} + (i_{L0} - i_{L\infty}) \times e^{-\frac{t}{\tau}}\)
                \item \(v_C (t) = v_{C \infty} + (v_{C 0} - v_{C \infty}) \times e^{-\frac{t}{\tau}}\)
            \end{itemize}

            \vfill\null
            \columnbreak

            Rampa
            \begin{itemize}
                \item \(i_L (t) = \frac{V_L}{L} (t - T_0) + I_{L0}\)
                \item \(v_C (t) = \frac{I_C}{C} (t - T_0) + V_{C0}\)
            \end{itemize}

            \vfill\null
            \columnbreak

            \[
                \tau_L = LG = \frac{L}{R}
                \qquad\qquad
                \tau_C = RC
            \]

        \end{multicols}

    \vspace{-3\baselineskip}
    \subsection*{Circuiti magnetici}
    \vspace{-1.45\baselineskip}

        \begin{gather*}
            \Psi_B \text{: flusso attraveso la sezione}, \, \Phi_B \text{: flusso concatenato con l'avvolgimento} \quad \Phi_B (t) = A(t) \times B(t) \qquad v = L \, \dv{i}{t}, \, \Phi_B = L I \rightarrow v = \dv{\Phi_B}{t}\\
            \Phi_B = n \, \Psi_B \qquad \mathcal{R} = \frac{l}{S \mu} \qquad \mu \text{: permeabilità magnetica} \qquad \mathcal{F} = \mathcal{R} \, \Psi_B \qquad \text{Equivalente ad un generatore di tensione con } V_{Eq} = n \, I
        \end{gather*}

    \vspace{-1.55\baselineskip}
    \subsection*{Induzione magnetica}
    \vspace{-.5\baselineskip}

        \begin{minipage}[t]{.4\textwidth}
            \vspace{-2cm}

            \begin{enumerate}
                \item Prendo il verso della sorgente come positivo
                \item LKT a sinistra dell'uguale\\(verso: regola della mano destra)
                \item Alla destra dell'uguale \(\rightarrow + \dv{\Phi}{t}\)
            \end{enumerate}

        \end{minipage}
        \hspace{.6cm}
        \begin{minipage}[t]{.1\textwidth}
            \vspace{-1.75cm}

            Esempio: 
            \[
                E - R \, I = \dv{\Phi}{t}
            \]

        \end{minipage}
        \hspace{.6cm}
        \begin{minipage}[t]{.1\textwidth}

            \begin{circuitikz}[scale=.75, every node/.style={scale=.75}]
    \draw
    (0,0) to[generic, R=\(R\)] (2,0)
        to[V, american, V=\(E\)] (2,-2)
        to[short] (0,-2)
        to[short, i<=\(I\)] (0,0)
    ;
\end{circuitikz}

        \end{minipage}
        \hspace{.6cm}
        \begin{minipage}[t]{.3\textwidth}
            \vspace{-2cm}

            \subsubsection*{Carica di un condensatore}

                \[
                    W(t_0, t_1) = \frac{1}{2} C \left[v^2 (t_1) - v^2 (t_0)\right] \; \unit{J}
                \]

            \vspace{-.5cm}

            \subsubsection*{Carica di un induttore}

                \[
                    W(t_0, t_1) = \frac{1}{2} L \left[i^2 (t_1) - i^2 (t_0)\right] \; \unit{J}
                \]

        \end{minipage}

\end{document}